% %!TEX root = ../username.tex
% \chapter{In the beginning: Knuth said ``Let there be \TeX''}\label{text}
% Now that I've tried to convince you that \lt is going to be better than \msw for your IS, you're saying, ``So how do I use it?'' Well let's start with some basic things. First, how is a document structured in \LaTeX?

% A \emph{document} for \lt is all the stuff that comes between the \verb|\begin{document}| and \verb|\end{document}| tags. The\verb| username.tex| file has the \verb|\begin{document}| and \verb|\end{document}| tags. ``OK, but how do I get my chapters to print?'' You save the chapters in the \verb|chapters| folder and put an \verb|\include{chapters/chaptername}| command in \verb|username.tex| after the \verb|\begin{document}| and before the
% \verb|\end{document}| tag. \verb|username.tex| already has some examples of including chapters; you can just alter them to have your chapter names. I should also mention that the \% symbol is used for comments. The \verb|username.tex| file has several comments that are intended for you and try to explain what is happening. Oh, and if you need a \% symbol enter \verb+\%+.

% Now to write your first chapter. I would recommend saving this chapter (chapter1.tex) under a different name and making changes to the new copy. The most basic structural elements that you need to know are the paragraph, \ic{chapter}, \ic{section}, and \ic{subsection}. A new paragraph is obtained by putting a blank line in the source file.  The other commands are very easy to use. If I want to start a new section I enter \texttt{$\backslash$section[My new section]\{An example of making a new section and giving it a short name\}} (the part in square brackets is optional) and get

% \section[My new section]{An example of making a new section and giving it a short name}\label{sec:newsec}

% The \ic{chapter} and \ic{subsection} commands work in the same manner. Each new chapter must have \texttt{$\backslash$chapter[short name]\{chapter name\}} as its first line.

% ``Hey, wait a minute. What if I need to refer to that section? How can I do that?'' It's actually as simple as adding\verb+\label{labelname}+ at the end of the \ic{chapter} command like\texttt{$\backslash$section[My new section]\{An example of making a new section and giving it a short name\}$\backslash$label\{sec:newsec\}}. Now I can refer to Section \ref{sec:newsec} by typing \verb+\ref{sec:newsec}+. You can label just about anything and refer to the label to get an automatically generated number for the item. This means that you need to come up with a labeling scheme before you start writing and stick with it.

% Some other things you'll need to be able to do include italicizing and bolding text and creating lists. These are also easy to accomplish. For example, I can use \ic{emph} or \ic{textit} to italicize text. To italicize homework, I would enter \verb|\emph{homework}| or \verb|\textit{homework}| to produce \textit{homework}. To obtain \textbf{bold} text you would use the \ic{textbf} command. And what about lists?

% There are several kinds of lists\index{lists} (enumerated, itemized, and descriptive) and each has its own place and environment. An enumerated\index{lists!enumerated} list is good for outlining or ordered lists:

% \begin{singlespace}
% \begin{example}
% \begin{enumerate}
% \item First main idea
% \begin{enumerate}
% \item First subpoint
% \item\label{enum:1b} Second subpoint
% \end{enumerate}
% \item Second main idea
% \end{enumerate}
% \end{example}
% \end{singlespace}

% The itemized\index{lists!itemized} list is good for unordered lists or bullet points:

% \begin{singlespace}
% \begin{example}
% \begin{itemize}
% \item Idea
% \item Idea
% \item Idea
% \item Idea
% \end{itemize}
% \end{example}
% \end{singlespace}

% And the descriptive\index{lists!descriptive} list is good for definitions; however, \ip{amsthm} already has a definition environment, and you will most likely not need the description environment. In any event, here is an example:

% \begin{singlespace}
% \begin{example}
% \begin{description}
% \item[First item:] Idea
% \item[Second item:] Idea
% \item[Third item:] Idea
% \end{description}
% \end{example}
% \end{singlespace}

% Notice the use of brackets in the last example. The brackets are optional and the text in the brackets is used as the label for the item. You should also note that you can label an item for later reference see \ref{enum:1b}. There are several options for changing the format of the list environments and a package, \ip{paralist}, for customizing lists which are described in section 3.3 of \citet{mgbcr04}.

% \section{Theorems, definitions, examples, oh my!}
% The next thing you'll probably need to do is enter definitions, theorems, and examples. Below you will find some examples. On the left you will see the text typed into the document and on the right what it looks like when formatted. These examples are intended to give you a sense of what type of mathematical expressions \lt handles. You should look at Appendix~\ref{math} for a more complete discussion of entering mathematics. In the beginning you will not know all the commands that you need to enter. Don't worry. Each of the suggested editors has a palette that shows you a picture of what you want and puts the correct commands into the document when you click the picture. As you look at these examples, keep it in mind that some of them use some user defined commands which can be found in \verb|styles/personal.tex|. Now let's look at Definition~\ref{def1}, Theorem~\ref{introwatthm}, and equation~\ref{m.1diasumtwo}.

% \begin{singlespace}
% \begin{example}
% \begin{defn}[One of Ramanujan's
%  third order mock theta 
%  functions]\label{def1}
%  \begin{equation}\label{introf(q)} 
%  f(q)=1+\sum_{y=1}^{\infty}
%  \frac{q^{y^2}}{(1+q)^2(1+q^2)^2
%  \cdots (1+q^y)^2}.
%  \end{equation}\end{defn}
% \end{example}
% \end{singlespace}

% \begin{singlespace}
% \begin{example}
% \begin{thm}[Watson's 
% transformation of 
% $f(q)$]\label{introwatthm}
% \begin{equation}\label{introf}
% \qrfac{q}{\infty}
% \sum_{y=0}^{\infty} q^{y^2}
%  \qrfac[-2]{- q}{y}=1+
%  \sum_{y=1}^{\infty}
%  \frac{(-1)^{y}
%  4q^{(3/2)y^2+
%  (1/2)y}}{(1+q^{y})}.
%  \end{equation}\end{thm}
% \end{example}
% \end{singlespace}

% This is a more complicated example which uses the \ic{substack} command to have multiple summation criteria.
% \begin{singlespace}
% \begin{example}
% \begin{align}\label{m.1diasumtwo}
% \left[NUM\right]_1^{(\fl)}(q;b;
% \bvec{x})=&\ q\sum\limits_{
% \substack{ 0\leq r,t 
% \leq\fl-1}}
% q^{r+t}\sum\limits_
%  {\substack{{\lambda
%  \vdash (r+t)}\\
%   \lambda/1^r\in V_t\\
%   \ell(\lambda)\leq \fl-1}}
%   \mathrm{s}_{(b,\lambda)}
%   (\bvec{x}).\end{align}
% \end{example}
% \end{singlespace}

% Another thing that one might need to do is create piecewise definitions. This can be accomplished by using the \verb|cases| \index{cases@\verb+cases+} environment. This example also uses the \ic{intertext} command to put text between displayed equations.
% \begin{singlespace}
% \begin{example}\begin{subequations}\label{2c1BP}
% \begin{alignat}{2}\label{2c1BPa} 
% A_{y_1}:=&\begin{cases}
%  1 &\text{for $y_1=0$},\\
% \frac{-1)^{y_1}
% 4q^{y_1}q^{\binom{y_1}{2}}}
% {\qrfac{q}{2y_1}(1+q^{y_1})}
% &\text{for $y_1>0$}\end{cases}\\
% \intertext{and} B_{y_1}:=&
% \qrfac[-1]{-q}{y_1}\qrfac[-1]
% {-q}{y_1}=\qrfac[-2]{-q}{y_1}
% &.\label{2c1BPb}\end{alignat}
% \end{subequations}
% \end{example}
% \end{singlespace}

% Finally, if you need to incorporate examples into your thesis you can do it using the example environment, as seen in Example~\ref{ex:ex}.
% \begin{singlespace}
% \begin{example}
% \begin{ex}[An example example]
% \label{ex:ex}
% This is an example of including an
%  example. Kind of silly isn't it.
%  \end{ex}
% \end{example}
% \end{singlespace}

% \section{Putting code in the main body of the thesis}
% There is one last textual item which Computer Science majors and probably some Mathematics majors will need to incorporate, pseudocode\index{pseudocode}. To do this I would suggest using the \ic{lstlisting} environment. Below is an example set up for the \ip{listings} package. You could put your modifications to this set up into the \texttt{personal.tex} file in the \texttt{styles} folder. Documentation on the \ip{listings} package can be found in the \texttt{doc} folder with the documentation for the other packages.
% \lstset{
%                language =Pascal, % pick a language style
%                emph={return,natural, numbers, integers, increasing},
%                emphstyle={\bfseries},% choose other keywords and a format
%                linewidth=.95\textwidth, breaklines=true, commentstyle=\textit,
%                stringstyle=\upshape, showspaces=false, numbers=left,
%                numberstyle=\tiny, basicstyle=\small, xleftmargin=30pt,
%                breakautoindent=true, captionpos=b
%                }
% {\small\begin{singlespace}
% \begin{verbatim}
% \lstset{
%         language =Pascal, % pick a language style
%         emph={return,natural, numbers, integers, increasing},
%         emphstyle={\bfseries},% choose other keywords and a format
%         linewidth=.95{\textwidth}, breaklines=true,commentstyle=\textit,
%         stringstyle=\upshape,showspaces=false,numbers=left,
%         numberstyle=\tiny,basicstyle=\small,xleftmargin=30pt,
%         breakautoindent=true,captionpos=b
%         }
% \end{verbatim}
% \end{singlespace}}

% The listing in Listing~\ref{largesteven} gives an algorithm for finding the largest even integer in a given list of $n$ integers. I have used the \texttt{mathescape}\index{listings!mathescape} option to be able to incorporate mathematics in the listing. The actual code put in the thesis is given first and the formatted output follows.

% {\small\begin{singlespace}
% \begin{verbatim}
% \begin{lstlisting}[mathescape, caption= Find the location 
% of the largest even integer in a list,label=largesteven]
% procedure $largestevenlocation$($a_1, a_2, \ldots, a_n$: integers)
% $k$:=0
% $largest$:=-$\infty$
% for $i$:=1 to $n$
%   if ($a_i$ is even and $a_i>largest$) then
%   begin
%     $k$:=$i$
%     $largest$:=$a_i$
%   end
% end
% return $k$
% \end{lstlisting}
% \end{verbatim}
% \end{singlespace}
% }
% \begin{singlespace}
% \begin{lstlisting}[mathescape, caption= Find the location
%  of the largest even integer in a list,label=largesteven]
% procedure $largestevenlocation$($a_1, a_2, \ldots, a_n$: integers)
% $k$:=0
% $largest$:=-$\infty$
% for $i$:=1 to $n$
%   if ($a_i$ is even and $a_i>largest$) then
%   begin
%     $k$:=$i$
%     $largest$:=$a_i$
%   end
% end
% return $k$
% \end{lstlisting}
% \end{singlespace}
% The code in Listing~\ref{quartsearch} is an improvement on Binary search. The algorithm reduces the size of the search by a factor of four at each iteration. It provides another example of using the \ic{lstlisting} environment.
% \begin{singlespace}\small
% \begin{verbatim}
% \begin{lstlisting}[mathescape,caption=Quartary search,
% label=quartsearch]
% procedure $quartarysearch$($x$: integer, $a_1, a_2,
%  \ldots, a_n$: increasing integers)
% $i$:=$1$
% $j$:=$n$
% while $i<j-2$
% begin
%   $l:=\lfloor(i+j)/4\rfloor$
%   $m:=\lfloor(i+j)/2\rfloor$
%   $u:=\lfloor3(i+j)/4\rfloor$
%   if $x>a_m$ then
%     if $x\leq a_u$ then
%     begin
%       $i:=m+1$
%       $j:=u$
%     end
%     else
%      $i:=u+1$
%   else if $x>a_l$ then
%     begin
%       $i:=l+1$
%       $j:=m$
%     end
%     else $j:=l$
% end
% if $x=a_i$ then $location:= i$
% else if $x=a_j$ then $location:= j$
% else if $x=a_{\lfloor(i+j)/2\rfloor}$ then
%  $location:= \lfloor(i+j)/2\rfloor$
% else $location:= 0$
% return $location$
% \end{lstlisting}
% \end{verbatim}
% \end{singlespace}
% \begin{singlespace}
% \begin{lstlisting}[mathescape,caption=Quartary search,label=quartsearch]
% procedure $quartarysearch$($x$: integer, $a_1, a_2, \ldots, a_n$: increasing integers)
% $i$:=$1$
% $j$:=$n$
% while $i<j-2$
% begin
%   $l:=\lfloor(i+j)/4\rfloor$
%   $m:=\lfloor(i+j)/2\rfloor$
%   $u:=\lfloor3(i+j)/4\rfloor$
%   if $x>a_m$ then
%     if $x\leq a_u$ then
%     begin
%       $i:=m+1$
%       $j:=u$
%     end
%     else
%      $i:=u+1$
%   else if $x>a_l$ then
%     begin
%       $i:=l+1$
%       $j:=m$
%     end
%     else $j:=l$
% end
% if $x=a_i$ then $location:= i$
% else if $x=a_j$ then $location:= j$
% else if $x=a_{\lfloor(i+j)/2\rfloor}$ then $location:= \lfloor(i+j)/2\rfloor$
% else $location:= 0$
% return $location$
% \end{lstlisting}
% \end{singlespace}

\chapter{Innovative Technologies in Personalized Fashion}
\label{chap:Chaptert2}


\section{Wardrobe Management in Fashion}
Managing personal wardrobes is essential for effective styling, enabling individuals to organize their clothing to enhance usability and creativity in outfit selection. Traditional wardrobe organization methods, such as sorting by type, color, or season, have long been relied upon to maintain order. However, as wardrobes expand, these manual techniques often become cumbersome, leading to underutilized or forgotten garments. Studies on wardrobe utilization emphasize that the proportion of clothing items worn relative to the total wardrobe size directly impacts how efficiently individuals engage with their clothing collections \cite{choo14}.

To address these challenges, digital wardrobe management tools have emerged, offering users the ability to catalog their clothing collections and visualize outfit combinations on mobile platforms. Research on fashion technology and mobile applications highlights how digital platforms have transformed wardrobe organization, making it easier for individuals to manage their clothing and plan outfits \cite{nie2013between}. Unlike physical methods of wardrobe organization, mobile applications allow users to quickly search, categorize, and retrieve clothing items, enhancing wardrobe accessibility.

Beyond organization, studies indicate that users who creatively mix and match their clothing achieve greater wardrobe utilization and experience higher levels of use innovativeness—a trait associated with sustainable fashion practices \cite{choo14}. This suggests that wardrobe management tools should not only function as digital closets but also encourage users to maximize their existing clothing through features such as outfit creation and styling recommendations. By enabling users to experiment with outfit combinations, digital wardrobe solutions can reduce clothing waste and promote sustainable consumption habits.

Despite the growing presence of wardrobe cataloging applications, many existing tools lack essential functionalities such as cross-device synchronization, outfit planning, and real-time data storage \cite{nie2013between}. Many apps focus on basic cataloging features but do not integrate advanced tracking mechanisms that would allow users to analyze their wardrobe usage patterns. As the intersection of fashion and technology continues to evolve, there is a growing demand for wardrobe management solutions that support dynamic organization, outfit experimentation, and sustainability-focused engagement. By incorporating features that help users visualize, track, and interact with their wardrobe in a structured way, digital wardrobe applications can play a significant role in reducing overconsumption and improving clothing utilization in modern fashion culture.

\section{Fashion and Mobile Applications}
The fusion of fashion and mobile technology has grown substantially in recent years, propelled by the widespread use of smartphones and the demand for personalized, user-friendly digital experiences. Fashion-tech applications encompass a variety of tools, from e-commerce platforms facilitating online shopping to styling apps offering outfit ideas and trend exploration \cite{nie2013between}. However, most existing fashion apps concentrate on retail or style inspiration, with relatively few addressing the practical aspects of wardrobe management and the integration of fashion into users' daily routines \cite{fits2024}.

Research into fashion technology highlights the advantages of mobile applications in enhancing user engagement through visual aids, interactive interfaces, and personalized recommendations \cite{nie2013between}. Features such as virtual try-ons, outfit suggestions, and gesture-based interactions have been shown to improve user satisfaction by providing intuitive and unique ways to engage with fashion. Additionally, applications that assist users in combining and adapting fashion items to their personal style foster a stronger connection between individuals and the fashion system \cite{nie2013between}.

User satisfaction and engagement are critical to the success of fashion applications. Studies indicate that key factors such as information quality, system reliability, and service quality significantly influence how users perceive and interact with mobile fashion apps. Among these, system quality, including easy navigation and app responsiveness, plays the most substantial role in user satisfaction \cite{trivedi2018fashion}. Furthermore, personalization has been identified as a crucial moderating factor, enhancing the relationship between app quality and user satisfaction. Apps that tailor recommendations, offer customized outfit suggestions, and adapt to individual user preferences foster stronger user engagement and increased retention rates \cite{trivedi2018fashion}.

Although many apps focus on shopping and styling, few offer full wardrobe management features. Many existing apps lack robust functionality for organizing extensive clothing collections, creating complete outfits, or offering suggestions based on user preferences or weather conditions. A review of popular closet and outfit planning apps highlights this gap, noting that while some apps focus on outfit creation (e.g., Combyne), others provide wardrobe organization but impose limitations such as restricted item entries or manual background removal processes, which can be time-consuming \cite{fits2024}. Additionally, some long-standing wardrobe apps, such as Stylebook and Smart Closet, suffer from a lack of updates, while others, like GetWardrobe, prioritize analytics over intuitive usability \cite{fits2024}. These limitations reinforce the need for a wardrobe management solution that fully integrates outfit creation, accessibility, and ease of use.

Fashion applications are also evolving to facilitate two-way communication between developers and users. Platforms incorporating rating systems, user reviews, and prompt developer responses create a more interactive experience, enabling continuous improvement of functionalities and user satisfaction \cite{nie2013between}. As fashion further integrates with mobile technology, the development of wardrobe management solutions that not only improve organization but also support sustainable fashion practices remains a crucial area for innovation.


\section{Cloud Storage and Data Management in Mobile Apps}
Cloud storage is a cornerstone of modern mobile applications, enabling the management of substantial volumes of user data, including images, videos, and documents. It offers critical functionalities such as cross-device accessibility and real-time synchronization, ensuring users can easily access and update their data across various devices \cite{bai2016cloud}. This capability is particularly valuable for applications like wardrobe management tools, which require efficient storage, retrieval, and organization of clothing images and associated metadata.

In the area of mobile applications, platforms like Firebase provide scalable solutions that integrate real-time database functionality with secure cloud storage \cite{firebasecookbook}. Firebase supports dynamic updates and synchronization, ensuring that applications can deliver a consistent user experience, especially when managing rapidly changing data such as images and their metadata. However, Firebase is just one example; the broader field of cloud storage offers various tools and services that can be tailored to meet the specific needs of different applications \cite{firebase_intro}.

Cloud storage also presents challenges, including data security, latency issues, and storage limitations. Protecting sensitive user information necessitates strong encryption and secure authentication mechanisms, while efficient optimization techniques, such as incremental file synchronization, can help mitigate latency and bandwidth constraints \cite{bai2016cloud}. These challenges are particularly pertinent to applications handling high-resolution images or frequent data updates, as they demand both reliability and scalability to ensure a smooth user experience.

By utilizing modern cloud storage solutions, including Firebase, wardrobe management applications can achieve efficient data handling and enhanced accessibility. These systems are crucial in creating a complete and dynamic user experiences, enabling users to interact with their digital wardrobes in practical and efficient ways \cite{firebasecookbook}.
 
\section{Implications for My Digital Wardrobe}
The insights from existing technologies and their applications have directly influenced the design and development of My Digital Wardrobe. Recognizing the limitations of traditional wardrobe management methods and the gaps in current digital solutions, this application aims to provide a comprehensive platform that addresses these challenges.

My Digital Wardrobe incorporates features that encourage creative outfit planning and efficient clothing utilization, aligning with the concept of enhancing wardrobe utilization. By facilitating easy mixing and matching of items, the app promotes innovative use of existing garments, contributing to more sustainable fashion habits.

Understanding the importance of simplifying user experience, the app integrates cloud storage solutions to ensure real-time synchronization and cross-device accessibility. This integration allows users to manage their wardrobe effortlessly, regardless of the device they use, addressing the need for dynamic and accessible wardrobe management tools.

Moreover, the app's design emphasizes an intuitive user interface, drawing from research that highlights the role of interactive and personalized features in enhancing user engagement. By focusing on user-friendly navigation and personalized recommendations, My Digital Wardrobe seeks to bridge the existing gap in wardrobe management solutions, offering a tool that is both practical and engaging for users.

In summary, the development of My Digital Wardrobe is informed by a thorough understanding of current technologies and their applications in fashion-tech. By addressing the identified gaps and challenges, the app aims to redefine personal wardrobe management through innovative and user-centered design. With these foundational insights in mind, the next chapter will focus on the development environment used to build the application, detailing the tools, frameworks, and technologies that enable the implementation of its key features.
