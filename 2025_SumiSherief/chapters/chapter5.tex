%!TEX root = ../username.tex
\chapter{Conclusion}
\label{chap:Chaptert6}
The development of \textit{My Digital Wardrobe} has focused on creating an accessible and efficient tool for wardrobe management. The app provides users with a structured way to upload, categorize, and explore their clothing collections, ensuring a more organized approach to outfit planning. Throughout the development process, several challenges emerged, particularly in balancing functionality with ease of use. While the app successfully integrates cloud storage and real-time accessibility, certain design limitations and missing features highlight opportunities for future improvement.

\section{Design Principles and Challenges}
A key design goal for My Digital Wardrobe was to create a clean, user-friendly interface that simplifies wardrobe organization. The login and upload screens were developed with a focus on clarity, using well-defined action buttons and recognizable icons to ensure intuitive navigation. The app follows a minimalist design, allowing users to upload and browse clothing items easily. However, some areas of the app feel visually cluttered, particularly the Wardrobe Screen and Outfit Screen, where multiple clothing items are displayed at once. While this does not significantly impact usability, refining the layout to improve spacing and readability will enhance the overall experience.

One of the most notable limitations is categorization. The current system requires users to manually assign clothing items to predefined categories, which are currently limited to tops and bottoms. This restriction reduces the app’s effectiveness as a wardrobe management tool, as users with diverse clothing collections cannot fully organize their wardrobe. Expanding the categorization system to include coats, jackets, dresses, accessories, and seasonal items is a necessary next step in making the app more comprehensive.

User testing was conducted using Expo, allowing the app to be deployed on different devices without extensive setup. This made testing quick and efficient, ensuring that the app could be evaluated across multiple platforms without delays. Testing confirmed that the app’s core features function as intended, with smooth navigation and a responsive interface. To simulate real wardrobe items, three tops and three bottoms were selected from Pinterest, a platform where users browse and save images, including fashion inspiration. Using Pinterest images helped create a realistic testing scenario, reflecting how users might upload and organize their own clothing collections.

While the fundamental features work well, personalization remains an area for improvement. Currently, the app provides a structured way to manage clothing, but it does not adapt to individual user preferences. Future updates will introduce customization features, such as the ability to tag favorite items, track frequently worn outfits, and receive outfit recommendations based on past selections. These enhancements will help tailor the experience to each user’s unique style, making the app more engaging and practical as a daily styling tool.

By addressing these challenges, My Digital Wardrobe will continue to refine its design while expanding its capabilities, ensuring a more user-focused wardrobe management experience.

\section{Future Enhancements}
To further enhance the functionality of My Digital Wardrobe, several key improvements are planned, with a focus on expanding categorization, streamlining automation, enhancing personalization, and enabling multi-platform accessibility.

One of the top priorities is expanding the categorization system beyond basic classifications like tops and bottoms. Future updates will introduce coats, jackets, accessories, and seasonal items, providing a more structured way to organize a wider range of clothing.

To enhance this process, the app will integrate machine learning (ML) for automatic classification, eliminating the need for manual categorization. An ML model will be trained on diverse clothing images to accurately recognize item types, streamlining uploads and reducing user effort. This implementation will involve dataset training, optimizing classification accuracy, and ensuring seamless integration with Firebase.

Another planned enhancement is web deployment, making My Digital Wardrobe accessible beyond mobile devices. While the current version is built using React Native and Expo, the web version will require adapting the interface for larger screens and optimizing performance for desktop use. Ensuring that wardrobe data remains synchronized across both mobile and web platforms will be a key challenge, necessitating refinements in Firebase’s authentication and storage mechanisms.

Personalization will also be a major focus moving forward. The app will introduce customization features that allow users to tag favorite items, track frequently worn outfits, and set wardrobe preferences. Additionally, an outfit recommendation system will be explored, utilizing stored wardrobe data to suggest clothing combinations based on past selections and user-defined preferences. These features will enhance usability by tailoring the experience to individual styling habits rather than simply serving as a digital catalog.

With these enhancements, My Digital Wardrobe will evolve into a more intelligent, adaptable, and widely accessible wardrobe management tool. Expanding categorization, automating classification, and enabling web deployment will ensure that users have a full experience across devices while reducing manual effort. By refining personalization features and implementing outfit recommendations, the app will move beyond basic wardrobe organization, offering a more interactive and user-focused styling experience. As development progresses, the focus will remain on optimizing usability, streamlining functionality, and expanding platform compatibility.


