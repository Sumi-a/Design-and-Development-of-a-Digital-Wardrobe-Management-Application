% %!TEX root = ../username.tex

\chapter{Introduction}\label{intro}

\section{Background and Motivation}
Fashion is an essential form of self-expression, allowing individuals to communicate their personality and style through clothing. However, as wardrobes expand over time, managing and organizing clothing efficiently becomes increasingly difficult. People often struggle to keep track of their items, leading to underutilization, inefficient outfit planning, and unnecessary purchases. Traditional methods of wardrobe organization—such as physical sorting, manual cataloging, or relying on memory—are time-consuming and do not provide a practical way to visualize outfit combinations.

As more people rely on mobile apps for organization, digital wardrobe tools help users store, categorize, and plan outfits easily. My Digital Wardrobe, the app we build here, is designed as a mobile application that addresses these challenges by providing a digital space where users can upload images of their clothing, organize them into categories, and create outfits. By combining image storage, categorization, and outfit creation tools, the
 app allows users to efficiently manage their wardrobe and explore different outfit possibilities. The integration of cloud storage ensures that wardrobe data remains accessible across devices, and the app’s interface is designed for intuitive navigation, making it easy to browse clothing items and experiment with styling choices.

\section{Problem Statement}
The challenge of wardrobe management goes beyond simple organization. As wardrobes grow, it becomes difficult to maintain a clear overview of available clothing, leading to inefficiencies in outfit selection and wardrobe utilization. Existing fashion-related mobile applications often focus on e-commerce and shopping, but few are dedicated to helping users manage the clothing they already own. Without a structured digital solution, users may find it difficult to catalog their wardrobe, visualize outfit combinations, or plan what to wear efficiently.

My Digital Wardrobe fills this gap by providing a digital solution for wardrobe organization, allowing users to store, categorize, and access their clothing collection through an interactive interface. Unlike shopping-focused fashion apps, this application is designed to enhance the practical use of existing wardrobes, helping users make informed outfit choices without requiring them to purchase new items.


\section{Research Objectives}
This thesis focuses on the design and implementation of My Digital Wardrobe, with the primary goal of creating a mobile application that improves wardrobe organization and outfit planning. The application’s core functionalities include:
\begin{enumerate}
 \item \textbf{Image Uploading and Categorization}: Users can take or upload photos of their clothing, categorizing them into relevant groups for easy browsing.
 \item \textbf{Outfit Creation and Storage}: Users can mix and match clothing items within the app to create and save outfits, building a personal style archive.
\item \textbf{Cloud Integration}: By using Firebase for storage, the app ensures accessibility from any device, maintaining wardrobe data securely.
\item \textbf{User-Friendly Interface}: The design prioritizes smooth navigation, allowing users to quickly access clothing items, experiment with different outfit combinations, and manage their wardrobe effortlessly.
\end{enumerate}

The following chapters explore the technical design and implementation of My Digital Wardrobe, detailing how its core features were developed and integrated. Chapter \ref{chap:Chaptert2} provides a literature review on innovative technologies in personalized fashion, exploring existing research, applications, and technological trends that inform the app’s development. Chapter \ref{chap:Chaptert3} outlines the development environment, describing the tools, frameworks, and technologies used. Chapter \ref{chap:Chapter4} focuses on the system architecture and functionality, explaining how different components interact to enable wardrobe management. Chapter \ref{chap:Chaptert5} delves into design and implementation, providing a detailed breakdown of feature development, user interface design, and backend integration. Finally, Chapter \ref{chap:Chaptert6} presents the conclusion, summarizing key findings, challenges, and potential areas for future improvement.


